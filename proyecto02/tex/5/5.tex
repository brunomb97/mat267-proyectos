\documentclass[../main.tex]{subfiles}
\begin{document}
    \begin{enunciado}
        Considere el proceso $X_t = \delta + X_{t-1} + \epsilon_t$, donde $t = 1,2,...,\epsilon_t$ es una secuencia de variables aleatorias iid con media cero y varianza $\sigma^2$.
        \begin{enumerate}
            \item Escriba la ecuación del proceso $X_t$ como sigue
                \begin{align}
                    \label{5:eq}
                	X_t = \delta t + \SUM j 1 t \epsilon_j.
                \end{align}
        \end{enumerate}

    \end{enunciado}
    \begin{demostracion}
        Observe que $X_t$ puede escribirse como (\ref{5:eq}). En efecto,
            \begin{itemize}
            	\item Para $t = 1$, se tendrá $X_t = \delta + \epsilon_1$.
            	\item Suponga que la igualdad se tiene para $t - 1$. Luego para $t$ se tendrá
                    \begin{align*}
                    	X_t = \delta + X_{t-1} + \epsilon_t = \delta + \delta (t-1) + \SUM j 1 {t-1} \epsilon_j + \epsilon_t = \delta t + \SUM j 1 t \epsilon_j,
                    \end{align*}
            \end{itemize}
        probando la equivalencia de las ecuaciones.
    \end{demostracion}
    \vspace{1em}
    \begin{enunciado}
    	\begin{enumerate}
            \setcounter{enumi}{1}
    		\item Calcule $\f \mu t = \Esp {X_t}$ y $\f V t = \Var {X_t}$.
    	\end{enumerate}
    \end{enunciado}
    \begin{demostracion}
    	Observe que
    	\begin{align*}
    		\f \mu t = \Esp{X_t} = \Esp{\delta t} + \SUM j 1 t \Esp{\epsilon_j} = \delta t.
    	\end{align*}
    	Además,
    	\begin{align*}
    		\f V t = \Var{X_t} &= \Esp{X_t^2} - \Esp{X_t}^2\\
    		&= \Esp{\delta^2 t - 2 \delta t \SUM j 1 t \epsilon_j + \pare{\SUM j 1 t \epsilon_j}^2} - (\delta t)^2\\
    		&= \delta^2 t - 2\delta t \underbrace{\SUM j 1 t \Esp{\epsilon_j}}_{= 0} + \SUM i 1 t \SUM j 1 t \Esp{\epsilon_i \cdot \epsilon_j} - \delta^2 t^2.\\
    		&= \delta^2 t + \SUM j 1 t \Esp{\epsilon_j^2} + \underbrace{\SUM i 1 t \sum_{\substack{j=1\\i \neq j}}^t \Esp{\epsilon_i}\Esp{\epsilon_j}}_{\text{independencia}} - \delta^2 t^2.\\
    		&= \delta^2 t + \sigma^2 t - \delta^2 t^2.
    	\end{align*}
    \end{demostracion}
    \vspace{1em}
    \begin{enunciado}
    	\begin{enumerate}
            \setcounter{enumi}{2}
    		\item ¿Es el proceso $X_t$ débilmente estacionario?
    	\end{enumerate}
    \end{enunciado}
    \begin{demostracion}
        Se define un proceso débilmente estacionario como
    	\begin{definicion}
            Sea $\llave{Z_t : t \in T}$ un proceso de segundo orden. Se dice que el proceso es \textbf{débilmente estacionario} si su función de media es constante y la función de covarianza entre $Z_t$ y $Z_s$ depende solo de la diferencia $t-s$. Es decir,
                \begin{itemize}
                    \item $\Esp {Z_t} = \mu,\ \paratodo t \in T$.
                    \item $\f C {t,s} = \Cov{Z_t,Z_s} = C(t-s),\ \paratodo t,s \in T$.
                \end{itemize}
    	\end{definicion}
        Observe que la función de medias no es constante, por tanto no es débilmente estacionario. $\qed$
    \end{demostracion}
\end{document}
