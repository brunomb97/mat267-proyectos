\documentclass[../main.tex]{subfiles}
\begin{document}
    \begin{enunciado}
    	Sea $X_t$ un proceso intrínsecamente estacionario. El semivariograma de $X_t$ se define como
            \begin{align*}
            	\f {\gamma_X} h = \frac 1 2 \Esp{\pare{X_{t+h} - X_t}^2}.
            \end{align*}
        \begin{enumerate}
        	\item Si $X_t$ es un ruido blanco con varianza $\sigma^2$, calcule $\f {\gamma_X} h$.
        \end{enumerate}
    \end{enunciado}
    
    \begin{demostracion}
        Como $X_t$ es un ruido blanco, entonces $\Esp {X_t} = 0$ y, por tanto,
            \begin{align*}
            	\Var {X_t} = \Esp{X_t^2} - \Esp{X_t}^2 = \Esp{X_t^2},\quad \paratodo t.
            \end{align*}
        Luego,
            \begin{align*}
            	\f {\gamma_X} h &= \frac 1 2 \Esp{\pare{X_{t+h} - X_t}^2}\\
            	&= \frac 1 2 \Esp{X_{t+h}^2 - 2 X_{t+h}X_t - X_t^2}\\
            	&= \frac 1 2 \pare{\Esp{X_{t+h}^2} - 2 \Esp{X_{t+h}X_t} + \Esp{X_t^2}}\\
            	&= \frac 1 2 \pare{\sigma^2 - 2 \Cov {X_{t+h},X_t} + \sigma^2}\\
            	&= \begin{funcionporpartes}{cc}
                        \sigma^2, & \text{ si } h \neq 0,\\
                        0, & \text{si } h = 0.
            	   \end{funcionporpartes}
            \end{align*}

    \end{demostracion}
    \vspace{1em}
    \begin{enunciado}
    	\begin{enumerate}
    		\setcounter{enumi}{1}
    		\item Si $X_t = \beta_0 + \beta_1 t + \epsilon_t$, donde $\epsilon_t$ es un ruido blanco con varianza $\sigma^2$, calcule $\f {\gamma_X} h$.
    	\end{enumerate}
    \end{enunciado}
    
    \begin{demostracion}
    	Se tendrá que
            \begin{align*}
            	\f {\gamma_X} h &= \frac 1 2 \cdot \Esp{\pare{X_{t+h} - X_t}^2}\\
            	&= \frac 1 2 \cdot \Esp{\pare{(\beta_0 + \beta_1 (t+h) + \epsilon_{t+h}) - (\beta_0 + \beta_1 t + \epsilon_t)}^2}\\
            	&= \frac 1 2 \cdot \Esp{\pare{\beta_1 h + \epsilon_{t+h} - \epsilon_t}^2}\\
            	&= \frac 1 2 \cdot \Esp{\beta_1^2 h^2 + \epsilon^2_{t+h} + \epsilon^2_t + 2\beta_1 h \cdot \epsilon_{t+h} - 2\beta_1 h \cdot \epsilon_t - 2\epsilon_{t+h} \cdot \epsilon_{t}}\\
            	&= \frac 1 2 \cdot \pare{\beta_1^2 h^2 + 2 \sigma^2 - 2 \Cov{\epsilon_{t+h},\epsilon_t}}\\
            	&= \begin{funcionporpartes}{ll}
                        \frac{\beta_1^2 h^2}{2} + \sigma^2, & \text{ si } h \neq 0,\\
                        0, & \text{ si } h = 0. \qed
            	   \end{funcionporpartes}
            \end{align*}

    \end{demostracion}

\end{document}
