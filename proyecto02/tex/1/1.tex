\documentclass[../main.tex]{subfiles}
\begin{document}
    \begin{enunciado}
        Considere un modelo de la forma
            \begin{align}
            	\label{1:modelo}
            	Y_t = \beta_0 + \SUM i 1 6 \beta_i \f \cos {\frac{2\pi t}{T_i}} + \epsilon_t,
            \end{align}
        donde el proceso $\e_t$ es un ruido blanco con varianza $\sigma^2$ y $T_i$ son los periodos de la serie.
            \begin{enumerate}
            	\item Escriba este modelo en la forma $\boldsymbol{Y} = \boldsymbol{X} \boldsymbol{\beta} + \boldsymbol{e}$.
            \end{enumerate}
    \end{enunciado}
    
    \begin{demostracion}
        Defina
            \begin{align*}
                X_t &= \begin{matriz}{ccccccc}
                        1 & \f \cos {\frac{2\pi t}{T_1}} & \f \cos {\frac{2\pi t}{T_2}} & \f \cos {\frac{2\pi t}{T_3}} & \f \cos {\frac{2\pi t}{T_4}} & \f \cos {\frac{2\pi t}{T_5}} & \f \cos {\frac{2\pi t}{T_6}}
                      \end{matriz}\\
                \beta &= \begin{matriz}{ccccccc}
                        \beta_0 & \beta_1 & \beta_2 & \beta_3 & \beta_4 & \beta_5 & \beta_6
                      \end{matriz}^t
            \end{align*}

    \end{demostracion}
    \vspace{1em}
    \begin{enunciado}
    	\begin{enumerate}
            \setcounter{enumi}{1}
    		\item Explique cómo obtener estimaciones de $\beta_0, ..., \beta_6$ y $\sigma^2$.
    	\end{enumerate}

    \end{enunciado}
    
    \begin{demostracion}
    	
    \end{demostracion}
    \vspace{1em}
    
    \begin{enunciado}
    	\begin{enumerate}
            \setcounter{enumi}{2}
    		\item ¿Qué consideraciones hay que establecer para que el modelo (\ref{1:modelo}) incluya una tendencia cuadrática?
    	\end{enumerate}

    \end{enunciado}
    
    \begin{demostracion}
    	
    \end{demostracion}
    \vspace{1em}
\end{document}
