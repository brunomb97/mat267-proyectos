\documentclass[../main.tex]{subfiles}
\begin{document}
    \begin{enunciado}
    	Sea $\f {C_X} \cdot$ la función de covarianza asociada a un proceso de media nula. Si
            \begin{align}
                \label{1:hipotesis}
            	\f {C_X} t = \f {C_X} 0,
            \end{align}
        para algún $t > 0$. Demuestre que $\f {C_X} \cdot$ es periódica.
    \end{enunciado}

    \begin{demostracion}
        Observe que
            \begin{align*}
            	\f {\gamma_X} h &= \frac 1 2 \Esp{\pare{X_{t+h} - X_t}^2}\\
            	&= \frac 1 2 \pare{\Esp{X_{t+h}^2} - 2 \Esp{X_{t+h}X_t} + \Esp{X_t^2}}\\
            	&= \frac 1 2 \pare{\f {C_X} 0 - 2 \f {C_X} h + \f {C_X} 0}\\
            	&= \f {C_X} 0 - \f {C_X} h.
            \end{align*}
        Por (\ref{1:hipotesis}), se tendrá que $\f {C_X} t = \f {C_X} 0$ y, por tanto, $\f {\gamma_X} T = \f {C_X} 0 - \f {C_X} T = 0$.
    \end{demostracion}

\end{document}
