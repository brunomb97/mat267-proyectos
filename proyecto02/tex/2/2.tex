\documentclass[../main.tex]{subfiles}
\begin{document}
    \begin{enunciado}
        Sea $\llave{X_t : t \in T}$ un proceso estacional normal con función con media $\mu_X$ y función de autocovarianzas $\f {C_X} \cdot$. Definamos la serie no lineal
            \begin{align}
                \label{2:modelo}
                Y_t = \f {\exp} {X_t},\quad t \in T.
            \end{align}
        \begin{enumerate}
        	\item Exprese la media del proceso $Y_t$ en términos de $\mu_X$ y $\f {C_X} 0$.
        \end{enumerate}
    \end{enunciado}
    \begin{demostracion}
        Utilizando la función generadora de momentos dada por
            \begin{align*}
                \f {M_X} t := \Esp{e^{tX}},\ \paratodo t \in \R.
            \end{align*}
        Luego, la función generadora de momentos para $X \sim \f N {\mu,\sigma^2}$ estará dada por
            \begin{align*}
                \f {M_X} t = \f \exp {\mu t + \frac 1 2 \sigma^2 t^2}
            \end{align*}
        Entonces, como $X_t \sim \f N {\mu_X, \f {C_X} {0}}$
            \begin{align*}
                \f {\mu_Y} t = \Esp{Y_t} = \Esp {\f \exp {X_t}} = \f {M_X} 1 = \f \exp {{\mu_X} + \frac 1 2 \f {C_X}{0}}.
            \end{align*}
    \end{demostracion}
    \vspace{1em}
    \begin{enunciado}
    	\begin{enumerate}
    		\setcounter{enumi}{1}
        	\item Determine la función de autocovarianza de $Y_t$.
    	\end{enumerate}
    \end{enunciado}

    \begin{demostracion}
        Observe que
            \begin{align*}
            	\f {C_Y} h &= \Esp {Y_{t+h} \cdot Y_t} - \Esp{Y_{t+h}}\Esp{Y_t}\\
            	&= \Esp {\f \exp {X_{t+h}} \f \exp {X_t}} - \f \exp{\mu_X + \frac 1 2 \f {C_X} 0}^2\\
            	&= \Esp {\f \exp {X_{t+h} + X_t}} - \f \exp{2\mu_X + \f {C_X} 0}.
            \end{align*}
        Se tiene que la suma de distribuciones normales es una distribución normal. Por tanto,
            \begin{align*}
            	X_t + X_s \sim \f N {2 \mu_X, 2 \f {C_X} 0 + 2 \f {C_X} h}.
            \end{align*}
        De esta forma, su primer momento estará dado por
            \begin{align*}
            	\f {M_{X_{t+h}+X_t}} 1 = \Esp{\f \exp {X_{t+h}+X_t}} &= \f \exp {2 \mu_X + \frac 1 2 \pare{2 \f {C_X} 0 + 2 \f {C_X} h}}\\ &= \f \exp {2 \mu_X + \f {C_X} 0 + \f {C_X} h}.
            \end{align*}
        Por lo que finalmente se obtiene
            \begin{align*}
            	\f {C_Y} h &= \f \exp {2 \mu_X + \f {C_X} 0 + \f {C_X} h} - \f \exp{2\mu_X + \f {C_X} 0}\\
            	&= \f \exp{2 \mu_X + \f {C_X} 0} \pare{\f \exp {\f {C_X} h} - 1}. \qed
            \end{align*}

    \end{demostracion}

\end{document}
