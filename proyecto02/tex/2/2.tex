\documentclass[../main.tex]{subfiles}
\begin{document}
    \begin{enunciado}
        Sea $\llave{X_t : t \in T}$ un proceso estacional normal con función con media $\mu_X$ y función de autocovarianzas $\f {C_X} \cdot$. Definamos la serie no lineal
            \begin{align}
                \label{2:modelo}
                Y_t = \f {\exp} {X_t},\quad t \in T.
            \end{align}
        \begin{enumerate}
        	\item Exprese la media del proceso $Y_t$ en términos de $\mu_X$ y $\f C 0$.
        \end{enumerate}

    \end{enunciado}
    \begin{demostracion}
        Utilizando la función generadora de momentos dada por
            \begin{align*}
                \f {M_X} t := \Esp{e^{tX}},\ \paratodo t \in \R.
            \end{align*}
        Luego, la función generadora de momentos para $X \sim \f N {\mu,\sigma^2}$ estará dada por
            \begin{align*}
                \f {M_X} t = \f \exp {\mu t + \frac 1 2 \sigma^2 t^2}
            \end{align*}
        Entonces
            \begin{align*}
                \f {\mu_Y} t = \Esp{Y_t} = \Esp {\f \exp {X_t}} = \f {M_X} 1 = \f \exp {\f {\mu_X}{t} + \frac 1 2 \f {C_X}{0}}.
            \end{align*}

    \end{demostracion}
    \vspace{1em}
    \begin{enunciado}
    	\begin{enumerate}
    		\setcounter{enumi}{1}
        	\item Determine la función de autocovarianza de $Y_t$.
    	\end{enumerate}
    \end{enunciado}

    \begin{demostracion}
        test
    \end{demostracion}

\end{document}
